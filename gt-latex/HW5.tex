\documentclass{article}
\usepackage{xcolor}
\usepackage{enumitem}
\usepackage{amsthm}
\usepackage{amsmath}
\usepackage{amssymb}
\usepackage{enumitem}
\usepackage{environ}
\usepackage{amsmath}
\usepackage[
    top=    1.00in,
    bottom= 1.00in,
    left=   1.00in,
    right=  1.00in,
]{geometry}





\NewEnviron{answer}{%
  \par\noindent\vspace{10pt}%
  \textcolor{red}{\textbf{Answer}:}
  \par\noindent\textcolor{red}{\BODY}%
  \par\vspace{-0.5\baselineskip}\raggedleft\textcolor{red}{$\blacksquare$}%
  \par\vspace{10pt}
}


\author{Taiyun}
\title{MATH-3012-QHS \\ Homework 5}
\date{Professor Dr. Kalila Lehmann}

\begin{document}

\maketitle


\textbf{Note}: These problems are often a little challenging the first time around 
so there is an extra point available on this homework. That is, 24 points are 
available but you'll still be graded out of 23.

\noindent\rule{\textwidth}{0.4pt}

\textit{For each part, please simplify your answer to a single integer; you may use 
a calculator. If it is helpful, you may use your answer to any part in any 
subsequent part. Please remember to give some verbal indication of your 
process in addition to just doing the necessary calculations.}




\begin{enumerate}
  
  \item (6 points) Amal needs to sort the distinct numbers 1-8 into 3 different groups.
  \begin{enumerate}
    \item If there needs to be at least one number in each group, in how many ways could this be done?
    \begin{answer}
      This problem requires finding the number of ways to partition a set of 8 distinct elements into 3 non-empty, unlabeled subsets. This quantity is calculated using the Stirling number of the second kind, denoted as $S(n, k)$, where $n=8$ and $k=3$.

      The value of $S(8, 3)$ is determined by the formula:
      $$S(n, k) = \frac{1}{k!} \sum_{j=0}^{k} (-1)^{k-j} \binom{k}{j} j^n$$
      For this specific case, the calculation is performed as follows:
      \begin{align*}
        S(8, 3) &= \frac{1}{3!} \left[ \binom{3}{0}(-1)^{3-0} 0^8 + \binom{3}{1}(-1)^{3-1} 1^8 + \binom{3}{2}(-1)^{3-2} 2^8 + \binom{3}{3}(-1)^{3-3} 3^8 \right] \\
        &= \frac{1}{6} \left[ 0 + (3)(1)(1^8) + (3)(-1)(2^8) + (1)(1)(3^8) \right] \\
        &= \frac{1}{6} \left[ 3 - 3(256) + 6561 \right] \\
        &= \frac{1}{6} [3 - 768 + 6561] \\
        &= \frac{5796}{6} \\
        &= 966
      \end{align*}
      Thus, there are 966 ways to sort the numbers into 3 non-empty, indistinguishable groups.
      \newline\newline
      \textbf{Answer}: 966
    \end{answer}
    \item Suppose there needs to be at least one number in each group, and the groups are labeled $\mathrm{A}, \mathrm{B}$, and C. How many options are there now?
    \begin{answer}
      When the groups are labeled, the problem becomes finding the number of ways to partition 8 distinct elements into 3 non-empty, labeled subsets. This is equivalent to finding the number of surjective (onto) functions from a set of 8 elements to a set of 3 elements.

      This can be calculated by taking the result from part (a), which is for unlabeled groups, and multiplying it by the number of ways to assign the distinct labels to these groups. There are $3!$ ways to label the 3 groups.
      $$ \text{Number of ways} = S(8, 3) \times 3! $$
      Using the value of $S(8, 3)$ from the previous part:
      $$ \text{Number of ways} = 966 \times 6 = 5796 $$
      Alternatively, the principle of inclusion-exclusion can be applied directly. The total number of ways to place 8 distinct items into 3 distinct groups is $3^8$. From this, the cases where one or more groups are empty must be subtracted.
      \begin{align*}
        \text{Number of ways} &= \binom{3}{3}3^8 - \binom{3}{2}2^8 + \binom{3}{1}1^8 \\
        &= (1)(6561) - (3)(256) + (3)(1) \\
        &= 6561 - 768 + 3 \\
        &= 5796
      \end{align*}
      Both methods confirm the result.
      \newline\newline
      \textbf{Answer}: 5796
    \end{answer}
    \item How many options for sorting does Amal have if he now is allowed to sort the numbers into at most 3 groups, i.e., some of the 3 groups could be empty?
    \begin{answer}
      This problem asks for the total number of ways to sort 8 distinct numbers into 3 labeled groups, with no restrictions on the groups being non-empty. This implies that for each of the 8 distinct numbers, there are 3 independent choices of group to which it can be assigned.

      Since the choice for each of the 8 numbers is independent, the total number of options is found by multiplying the number of choices for each number.
      $$ \text{Total options} = 3 \times 3 \times 3 \times 3 \times 3 \times 3 \times 3 \times 3 = 3^8 $$
      The result of this calculation is:
      $$ 3^8 = 6561 $$
      This answer can be verified by summing the number of ways to have exactly one, two, or three non-empty labeled groups.
      \begin{itemize}
        \item Exactly 3 non-empty groups: $5796$ (from part b).
        \item Exactly 2 non-empty groups: $\binom{3}{2} \times (2^8 - \binom{2}{1}1^8) = 3 \times (256 - 2) = 3 \times 254 = 762$.
        \item Exactly 1 non-empty group: $\binom{3}{1} \times 1^8 = 3 \times 1 = 3$.
      \end{itemize}
      The total sum is $5796 + 762 + 3 = 6561$, which confirms the direct calculation.
      \newline\newline
      \textbf{Answer}: 6561
    \end{answer}
  \end{enumerate}



  \newpage

  \item (6 points) Amal needs to sort the distinct numbers 1-8 into 3 different groups.
  \begin{enumerate}
    \item If there needs to be at least one number in each group, in how many ways could this be done?
    \begin{answer}
      [Your answer here.]
    \end{answer}
    \item Suppose there needs to be at least one number in each group, and the groups are labeled $\mathrm{A}, \mathrm{B}$, and C. How many options are there now?
    \begin{answer}
      [Your answer here.]
    \end{answer}
    \item How many options for sorting does Amal have if he now is allowed to sort the numbers into at most 3 groups, i.e., some of the 3 groups could be empty?
    \begin{answer}
      [Your answer here.]
    \end{answer}
  \end{enumerate}


  \newpage

  \item (5 points) How many nine-digit sequences have each of the numbers 2, 5, and 8 appearing at least once? You do not need to simplify your answer.
  \begin{answer}
    [Your answer here.]
  \end{answer}

  \item (5 points) Find the number of integer solutions to the equation $A+B+C=15$, where
  $$
  \begin{aligned}
  & 0 \leq A \leq 6 \\
  & 0 \leq B \leq 11 \\
  & 2 \leq C \leq 7
  \end{aligned}
  $$
  \begin{answer}
    [Your answer here.]
  \end{answer}

  \item (5 points) In how many ways can the letters of the word ARRANGEMENT be arranged so that there are exactly two pairs of consecutive identical letters?
  \begin{answer}
    [Your answer here.]
  \end{answer}
\end{enumerate}

\end{document}