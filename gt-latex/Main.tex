\documentclass{article}
\usepackage{xcolor}
\usepackage{enumitem}
\usepackage{amsmath}
\usepackage[
    top=    1.00in,
    bottom= 1.00in,
    left=   1.00in,
    right=  1.00in,
]{geometry}



\author{Taiyun}
\title{MATH-3012-QHS \\ M2.2-Quiz}
\date{Professor Dr. Kalila Lehmann}

\begin{document}

\maketitle

\section*{Solutions}

Here is an example sentence. 

\begin{enumerate}
    \item Find the number of permutations of all the letters in "GEOLOGICAL". Use a calculator to simplify your answer to an integer.
    \par
    \textcolor{red}{The number of permutations is 453,600.
    \begin{itemize}
        \item The word has 10 letters in total.
        \item The repeated letters are G (2 times), O (2 times), and L (2 times).
        \item The formula for permutations with repetitions is:
        \[
        \frac{n!}{n_1! n_2! \dots n_k!}
        \]
        \item Calculation:
        \[
        \frac{10!}{2! \cdot 2! \cdot 2!} = \frac{3,628,800}{8} = 453,600
        \]
    \end{itemize}}

    \item How many length 5 permutations of the letters in BOTTLE are there?
    \par
    \textcolor{red}{The number of length 5 permutations is 360.
    \begin{itemize}
        \item The word has 6 letters, with 'T' repeated twice.
        \item Case 1: Permutations containing both 'T's. We choose 3 unique letters from the remaining 4 (B, O, L, E) and arrange them with the two 'T's.
        \[
        \binom{4}{3} \cdot \frac{5!}{2!} = 4 \cdot 60 = 240
        \]
        \item Case 2: Permutations containing only one 'T'. We choose 4 unique letters from the remaining 4 and arrange them with the single 'T'.
        \[
        \binom{4}{4} \cdot 5! = 1 \cdot 120 = 120
        \]
        \item Total permutations:
        \[
        240 + 120 = 360
        \]
    \end{itemize}}

    \item How many length 4 permutations of the letters in COMPUTER are there?
    \par
    \textcolor{red}{The number of length 4 permutations is 1,680.
    \begin{itemize}
        \item The word has 8 unique letters.
        \item We use the k-permutation formula:
        \[
        P(n,k) = \frac{n!}{(n-k)!}
        \]
        \item Calculation:
        \[
        P(8,4) = \frac{8!}{(8-4)!} = \frac{8!}{4!} = 8 \cdot 7 \cdot 6 \cdot 5 = 1,680
        \]
    \end{itemize}}

    \item How many permutations of the symbols in "Elephantine" are there? Case matters! Use a calculator to simplify your answer to an integer.
    \par
    \textcolor{red}{The number of permutations is 9,979,200.
    \begin{itemize}
        \item The word has 11 symbols.
        \item The repeated symbols are 'e' (2 times) and 'n' (2 times).
        \item Calculation:
        \[
        \frac{11!}{2! \cdot 2!} = \frac{39,916,800}{4} = 9,979,200
        \]
    \end{itemize}}
\end{enumerate}

\end{document}