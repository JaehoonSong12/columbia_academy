\documentclass{article}
\usepackage{xcolor}
\usepackage{enumitem}
\usepackage{amsmath}
\usepackage[
  top=    1.00in,
  bottom= 1.00in,
  left=   1.00in,
  right=  1.00in,
]{geometry}



\author{Jihoo Choi}
\title{Math Note}
\date{\today}

\begin{document}

\maketitle

\section*{Solutions}

Here is an example sentence. 


% Latex is great for typesetting math equations!!
\begin{enumerate}
  \item The table gives characteristics of a trigonometric function 
  $f$ for selected intervals of $\theta$. 
  Which of the following could define $f(\theta)$ ?
  \begin{center}
    \begin{tabular}{|l|l|l|}
    \hline$\theta$ & $\pi<\theta<\frac{3 \pi}{2}$ & $\frac{3 \pi}{2}<\theta<2 \pi$ \\
    \hline$f(\theta)$ & negative & positive \\
    \hline
    \end{tabular}
  \end{center}
  \begin{enumerate}
    \item $\cos \theta$ only
    \item $\sin \theta$ only
    \item $\cos \theta$ or $\sin \theta$
    \item Neither $\cos \theta$ nor $\sin \theta$
  \end{enumerate}


  \item If $\sin \theta=-\frac{1}{2}$ and $\theta$ is in Quadrant III, what is the value of $\cos \theta$ ?

\end{enumerate}

\end{document}